%
% File emnlp2015.tex
%
% Contact: daniele.pighin@gmail.com
%%
%% Based on the style files for ACL-2015, which were, in turn,
%% Based on the style files for ACL-2014, which were, in turn,
%% Based on the style files for ACL-2013, which were, in turn,
%% Based on the style files for ACL-2012, which were, in turn,
%% based on the style files for ACL-2011, which were, in turn, 
%% based on the style files for ACL-2010, which were, in turn, 
%% based on the style files for ACL-IJCNLP-2009, which were, in turn,
%% based on the style files for EACL-2009 and IJCNLP-2008...

%% Based on the style files for EACL 2006 by 
%%e.agirre@ehu.es or Sergi.Balari@uab.es
%% and that of ACL 08 by Joakim Nivre and Noah Smith

\documentclass[11pt,a4paper]{article}
\usepackage{acl2015}
\usepackage{times}
\usepackage{url}
\usepackage{latexsym}
\usepackage{lipsum}

%\setlength\titlebox{5cm}

% You can expand the titlebox if you need extra space
% to show all the authors. Please do not make the titlebox
% smaller than 5cm (the original size); we will check this
% in the camera-ready version and ask you to change it back.


\title{Report: Dialog Act Classification\\\textit{using Word Embeddings \& Acoustic Features}}

\author{Jens Beck \\
  {\tt jens.beckl@ims} \\\And
  Fabian Fey \\
  {\tt fabian.fey@ims} \\\And
  Richard Kollotzek \\
  {\tt richard.kollotzek@ims} \\}

\date{}

\begin{document}
\maketitle

\begin{abstract}

\end{abstract}

\section{Introduction}
The general task is to classify lexical and auditory speech into one of four predefined \textit{dialog act classes}. A \textit{dialog act}, in this context, represents informal information of how a dialog system should respond to a users input. The four provided classes are \textit{statement, opinion, question} and \textit{backchannel}. To solve this task we developed \textit{convolutional neural networks} (CNN) that use lexical and acoustic features. For the development and training of the systems a subset of the \textit{Switchboard Dialog Act Corpus} was used. In next chapters we discuss the development of the systems and subsequently to that the research question \textbf{INSERT HERE}.

\section{Data \& Data Preperation}
In this section we discuss the \textit{Switchboard Dialog Act Corpus} and the extraction of the lexical and acoustic features.

	\subsection{The Switchboard Dialog Act Corpus}
	The \textit{Switchboard Dialog Act Corpus}, from now on abbreviated as \textit{SwDA}, consists of recordings with corresponding transcripts and \textit{dialog act classes}. 

	\subsection{Data Preprocessing}

\section{Baseline Systems}

\section{Results}

\section{Research Question: None}

\section{Conclusion}

\bibliographystyle{plain}
\bibliography{bibliography}
\nocite{*}

\end{document}
